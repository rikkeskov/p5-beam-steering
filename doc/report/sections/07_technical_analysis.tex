\chapter{Technical Analysis}
The purpose of beam steering is to increase the antenna gain in a specific direction to achieve reduction in interference and to save power. This is enabled by having the main lobe of a radiation pattern of a directional antenna pointing towards the target of transmission and reception. Beam steering is purposeful for narrow directional beams. Beam steering can be performed using manual, mechanical or electronic with the main differences being type of implementation and increasing speed of change of directivity from manual to electronic~\cite{ieee_beam_steering}~\cite{ieee_microchip_beam_steering}. This chapter explores the properties of antennas and beam steering methods, in order to understand antenna beam steering.
%https://ieeexplore-ieee-org.zorac.aub.aau.dk/document/9726284
%https://ieeexplore-ieee-org.zorac.aub.aau.dk/document/8275895

\section{Fundamentals of Antennas}
In order to develop a beam steering device for antennas it is necessary to understand antennas and their properties. Propagation, polarization, radiation characteristics are all properties of antennas that can vary based on the type of antenna.

\subsection{Propagation In Free Space}
Propagation of radio waves can be described with Maxwell's equations using the spherical coordinate system $\left(r, \theta, \phi, t\right)$ for antennas. The Maxwell equations in differential form are as follows
\begin{equation}
    \begin{split}
        \nabla \times \textbf{E} & = - \frac{\partial }{\partial t} \textbf{B} \\
        \nabla \times \textbf{H} & = \textbf{J} + \frac{\partial }{\partial t} \textbf{D} \\
        \nabla \cdot \textbf{B} & = 0 \\
        \nabla \cdot \textbf{D} & = \rho
    \end{split}
\end{equation}

with \textbf{E} being the electric field with unit [\SI{}{\volt\per\meter}], \textbf{B} being induction [\SI{}{\tesla}], \textbf{H} is magnetic field [\SI{}{\ampere\per\meter}], \textbf{D} being dielectric displacement [\SI{}{\ampere\second\per\meter\squared}], \textbf{J} being the current density [\SI{}{\ampere\per\meter\squared}] and $\rho$ being electric charge density [\SI{}{\coulomb\per\meter\cubed}].

The electric field and the magnetic field are always connected; the electric field is created by the magnetic field and vice versa. The electric field and the magnetic field in spherical coordinates are illustrated on figure \ref{fig:em_field} below.
\begin{figure}[H]
    \centering
    \includegraphics[width=0.6\textwidth]{figures/em_polar_coordinates.png}
    \caption{Electromagnetic field around a small antenna in far field range visualised in the spherical coordinate system~\cite[p. 58]{maxwell_theory}.} \label{fig:em_field}
\end{figure}

As visualised on figure \ref{fig:em_field} the electric field only depends on the $\theta$-component and the magnetic field only on the $\phi$-component when in the far field. The vector \textbf{r} is direction of the observation point. The length of \textbf{r} is the distance to the observation point ie. distance between transmitting and receiving antenna~\cite[p. 59]{maxwell_theory}.

The Poynting vector describes the power density and direction of the Electromagnetic flux and is the cross product of the electric and magnetic field 
\begin{equation}
    \textbf{S} = \textbf{E} \times \textbf{H}
    \tagaddtext{[\si{\watt\per\meter\squared}]}
\end{equation}
which points in the same direction as the wave propagation~\cite[p. 3]{ant_beam_form}. 

\subsubsection{Multipath propagation}
Because electromagnetic waves can reflect on surrounding surfaces, be changed by condition of the transmission medium or be Doppler shifted due to movement of objects, receiver or transmitter, the signal that reaches the receiving antenna can have travelled other paths than the direct line-of-sight path from the transmitter. The received signal is a summation of all input signals regardless of phase angles, phase shifts or direction and therefore the received signal might be distorted~\cite[pp. 1-2]{itu_multipath}. 

\subsection{Polarization}
Different antenna designs have different radiation patterns and polarization. Table \ref{tab:antenna_types} lists a number of different antenna designs and their polarization
\begin{table}[H]
    \centering
    \begin{tabular}{l|l} 
        \textbf{Type} & \textbf{Polarization} \\
        \hline
        \hline
        Isotropic antenna & \\
        Dipole antenna & Linear \\
        Patch antenna & Linear, circular \\
        Horn antenna & Linear, cicular \\
        Helical antenna & Circular \\
    \end{tabular}
    \caption{Table showing polarization of some typical antenna designs~\cite[p. 11]{ant_beam_form}.}
    \label{tab:antenna_types}
\end{table}

Polarization describes the classification of the plane propagated wave, which is a electromagnetic wave that propagates with constant velocity in a specific direction. The direction of the propagation of the electromagnetic wave is always perpendicular to the direction of the electric field \textbf{E} and both are perpendicular to the direction of the magnetic field \textbf{H}. If the direction of the electric field is constant with time and position, the polarization of the propagated wave is classified as linear. If the direction of the electric field changes by rotating uniformly around the axis of the propagated wave, the wave has circular polarization. The polarization of the receiving and transmitting antennas affects how the signal is detected. The signal is best received when the polarization of the receiving antenna is the same as the polarization for the transmitting antenna. Mismatch in polarization will result is less received signal power or more signal noise~\cite[p. 82-84]{direct_energy}. 

\subsection{Radiation Characteristics}
The electromagnetic waves are radiated to the near field and then far field. The far field is mathematically described as the distance $r>R_2$, with $R_2$ defined as
\begin{equation} \label{eq:far_field}
    R_2 = \frac{2 D^2}{\lambda}
    \tagaddtext{[\si{\meter}]}
\end{equation}

with $D$ being the largest dimension of the antenna or antenna array and $\lambda$ being the wavelength of the carrier frequency~\cite[p. 4]{ant_beam_form}.

The radiation characteristics of an antenna can be described by the directivity, which is defined as the ratio of the maximum power density $S\left( \theta, \phi \right)_{max}$ radiated to the average power density $S\left( \theta, \phi \right)_{avg}$ radiated by an antenna. The directivity is unitless~\cite[p. 63]{direct_energy}. An isotropic antenna is a theoretical antenna which radiates homogeneously in all directions, meaning that the magnitude of the power density vector \textbf{S} at a distance vector \textbf{r} is constant
\begin{equation} \label{eq:isotropic_radiation}
    D\left( \theta, \phi \right) = \left| \frac{S \left(r, \theta, \phi \right)}{S_{max}} \right|=1
\end{equation}

It is this theoretical isotropic radiator that the gain of antennas are in respect to. The gain of a directive antenna in a certain direction is called the antenna gain $G$~\cite[p. 12]{ant_beam_form}. The directivity doesn't depend on the distance $r$ in the far field meaning that at the receiver antenna, the relation $r \gg  R_2$ is assumed. 

The total radiated power $P_{r}$ is found by the surface integral of the power density. Assuming a spherical surface, the total radiated power is described as
\begin{equation} \label{eq:surface_integral}
    P_{r} = \int_{0}^{2 \pi} \int_{0}^{\pi} S\left( \theta, \phi \right) r^2 \sin \theta \, d\theta \, d\phi
    \tagaddtext{[\si{\watt}]}
\end{equation}
And further, averaging the radiated power over every direction in the sphere gives the relation
\begin{equation} \label{eq:average_power}
    P_{avg} = \frac{P_{r}}{4 \pi r^2}
    \tagaddtext{[\si{\watt\per\meter\squared}]}
\end{equation}
Replacing the average power density $S\left( \theta, \phi \right)_{avg}$ by the average power in every direction $P_{avg}$, the directivity of an antenna $D\left(\theta, \phi\right)$ can be defined as 
\begin{equation} \label{eq:directivity}
    D\left(\theta, \phi\right) = 4 \pi r^2 \frac{S \left(\theta, \phi\right)_{max}}{P_{r}}
\end{equation}

with $S\left(\theta, \phi\right)_{max}$ being the maximum power density and $P_{r}$ being the total radiated power of the antenna.

The power of the source $P_{s,t}$ to a transmitting antenna might not equal the radiated power $P_{r,t}$ due to power loss $p_{l,t}$. Power loss can happen because of reflection loss in the input medium (typically cable), conductor loss and inductor loss. The efficiency of the transmitting antenna $\eta$ is described as the ratio of the radiated power to the sourced power 
\begin{equation} \label{eq:antenna_efficiency}
    \eta = \frac{P_{r,t}}{P_{s,t}} = \frac{P_{r,t}}{P_{r,t}+P_{l,t}}
\end{equation}

The gain $G \left( \theta, \phi \right)$ of the antenna is the effective directivity, meaning how well the receiving or transmitting antenna is able to convert, respectively, electromagnetic waves or power into the other. The gain of a transmitting antenna can be calculated as
\begin{equation} \label{eq:gain}
    G_t \left( \theta, \phi \right) = \eta  D_t \left(\theta, \phi\right) = 4 \pi r^2 \frac{S_t \left(\theta, \phi\right)_{max}}{P_{s,t}}
\end{equation}

or expressed in decibel with respect to the isotropic radiator~\cite[p. 10]{ant_beam_form}~\cite[pp. 1.8-1.10]{ant_eng_hk}.
\begin{equation} \label{eq:gain_dbi}
    g_{dBi} = 10 \log_{10}\left(G\right)
    \tagaddtext{[\si{\decibel}]}
\end{equation}

\subsubsection{Friis Transmission Equation}
The Friis transmission equation explains how the received power at a receiver antenna is related to the power of the transmitting antenna. The receiver antenna receives energy from the transmitting antenna and the effectiveness of this is described as the effective area $A_r\left( \theta, \phi \right)$ assuming that the antenna is placed in the origin of the spherical coordinate system. If the antenna has the property of reciprocity, the effective area and the gain of the receiver antenna is related by 
\begin{equation} \label{eq:effectivate_area}
    A_r \left( \theta, \phi \right) = \frac{\lambda^2}{4 \pi} G_r \left( \theta, \phi \right)
    \tagaddtext{[\si{\meter\squared}]}
\end{equation}

If the gain of the transmitting antenna $G_t$ is in the direction of the receiver antenna $G_r$ then the angular dependencies of the antenna properties can be surpressed. The power of the receiving antenna is equal to the power density $S_t \left(\theta, \phi\right)$ multiplied by the effective area of the receiver antenna $A_r$, expressed as
\begin{equation} \label{eq:receiver_power}
    P_r = S_t A_r
    \tagaddtext{[\si{\watt}]}
\end{equation} 

As previously mentioned the directivity of an antenna does not depend on the distance $r$ from the antenna, and likewise so with the power density $S_t$, so the value of $S_t$ is equal regardless of the distance from the antenna in the far field with respects to the angular dependencies. Substituting $S_t$ and $A_r$ in equation \ref{eq:receiver_power} for $S_t$ isolated in equation \ref{eq:gain} and $A_r$ from equation \ref{eq:effectivate_area} yields
\begin{equation} \label{eq:friis}
    \begin{split}
        P_r & = \frac{G_t P_t}{4 \pi r^2} \frac{\lambda^2 G_r}{4 \pi} \\
        & = G_t  G_r P_t \left( \frac{\lambda}{4 \pi r} \right)^2
        \tagaddtext{[\si{\watt}]}
    \end{split}
\end{equation} 

Also called Friis transmission equation~\cite[pp. 1.10-1.11]{ant_eng_hk}. $G_t$ is the gain of the transmitting antenna in the direction of the receiver and $G_r$ is the gain of the receiving antenna in the direction of the transmitter. The radiation characteristics of an antenna in the far field is called the antenna radiation pattern and will look different depending on the design of the antenna. The radiatation pattern is dependent on the anglular properties $\theta$ and $\phi$ and is usually visualised in a plane parallel to the electric field and called an \textbf{E} plane pattern or elevation plane pattern, or parallel to the magnetic field and called a \textbf{H} plane pattern or Azimuth plane pattern~\cite[p. 79-80]{direct_energy}\cite[p. 13-14]{ant_beam_form}.

\subsubsection{Antenna Design}
The design of the antenna affect the radiation characteristics, the bandwidth and the range of frequencies that the antenna is able to transmit or receive~\cite[p. 76]{direct_energy}. The antenna types described in table \ref{tab:antenna_types} are some of the typical design types along with the isotropic antenna which is a theoretical antenna~\cite[p. 11]{ant_beam_form}. 

A dipole antenna is two wires or rods pointing at the opposite direction of each other, for example towards positive and negative \textit{z} in the spherical coordinate system. In the Azimuth plane the radiation pattern is a circle centered at the centre of the dipole antenna, whereas in the elevation plane, the radiation pattern is two ears extending from the centre to either side (see figure \ref{fig:dipole_1} and \ref{fig:dipole_2})~\cite[pp. 12-14]{ant_beam_form}.
\begin{figure}[H]
    \begin{minipage}{0.45\textwidth}
        \centering
        \includegraphics[width=0.9\textwidth]{figures/farfield (f=2.4) dipole.png} % first figure itself
        \caption{Dipole antenna radiation pattern in the Azimuth plane.} 
        \label{fig:dipole_1}
    \end{minipage}\hfill
    \begin{minipage}{0.45\textwidth}
        \centering
        \includegraphics[width=0.9\textwidth]{figures/farfield (f=2.4) dipole_1.png} % second figure itself
        \caption{Dipole antenna radiation pattern in the elevation plane. Figures generated with \textit{CST Studio} dipole antenna example ($f=\SI{2.4}{\giga\hertz}$).}
        \label{fig:dipole_2}
    \end{minipage}
\end{figure}
The dipole antenna is good at ?? and bad at ?? and it can also be configured differently like a yagi antenna and ??
% p 12 antenna beam forming

A patch antenna ????% p 15 -||- ???
\begin{figure}[H]
    \begin{minipage}{0.45\textwidth}
        \centering
        \includegraphics[width=0.9\textwidth]{figures/farfield (f=2.4) patch.png} % first figure itself
        \caption{Patch antenna radiation pattern in the Azimuth plane.} 
        \label{fig:patch_1}
    \end{minipage}\hfill
    \begin{minipage}{0.45\textwidth}
        \centering
        \includegraphics[width=0.9\textwidth]{figures/farfield (f=2.4) patch_1.png} % second figure itself
        \caption{Patch antenna radiation pattern in the elevation plane. Figures generated with \textit{CST Studio} ($f=\SI{2.4}{\giga\hertz}$).}
        \label{fig:patch_2}
    \end{minipage}
\end{figure}

Horn antennas come in many shapes and sizes which affect the radiation pattern and gain but common amongst the different designs is high beam directivity~\cite[p. 14.1]{ant_eng_hk}. A rectangular horn antenna has a fan-shaped radiation pattern and the beamwidths of the fan can be changed by the varying the aperture dimensions ?????
\begin{figure}[H]
    \begin{minipage}{0.45\textwidth}
        \centering
        \includegraphics[width=0.9\textwidth]{figures/farfield (f=2.4) horn.png} % first figure itself
        \caption{Horn antenna radiation pattern in the Azimuth plane.} 
        \label{fig:horn_1}
    \end{minipage}\hfill
    \begin{minipage}{0.45\textwidth}
        \centering
        \includegraphics[width=0.9\textwidth]{figures/farfield (f=2.4) horn_1.png} % second figure itself
        \caption{Horn antenna radiation pattern in the elevation plane. Figures generated with \textit{CST Studio} horn antenna example ($f=\SI{2.4}{\giga\hertz}$).}
        \label{fig:horn_2}
    \end{minipage}
\end{figure}

A helical antenna ???% pdf p 346 engineering handbook (explain modes!) 
\begin{figure}[H]
    \begin{minipage}{0.45\textwidth}
        \centering
        \includegraphics[width=0.9\textwidth]{figures/farfield (f=2.4) helical.png} % first figure itself
        \caption{Helical antenna radiation pattern in the Azimuth plane.} 
        \label{fig:helical_1}
    \end{minipage}\hfill
    \begin{minipage}{0.45\textwidth}
        \centering
        \includegraphics[width=0.9\textwidth]{figures/farfield (f=2.4) helical_1.png} % second figure itself
        \caption{Helical antenna radiation pattern in the elevation plane. Figures generated with \textit{CST Studio} ($f=\SI{2.4}{\giga\hertz}$).}
        \label{fig:helical_2}
    \end{minipage}
\end{figure}

Antennas can also be put together in arrays. This changes the characteristics of the radiation pattern and ??

\todo[inline,color=red]{describe typical antenna designs. see p11 rf antenna beam forming}

%Horn antennas - engineering hanbook p. 24

% 

\section{Beam Steering Methods}
Beam steering can be done manually, mechanically or electrically.

\subsection{Manual and Mechanical Steering}
\todo[inline,color=green]{explain motor control and types of motors with type of control loops}
%https://eu.aspina-group.com/en/learning-zone/columns/what-is/017/

\subsection{Electrical Steering}