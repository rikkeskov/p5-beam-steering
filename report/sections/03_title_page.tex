\pdfbookmark[0]{English title page}{label:titlepage_en}
\aautitlepage{%
  \englishprojectinfo{
    Mechanical Antenna Beam Steering %title
  }{%
    Digital and Analogue Systems Interacting with the Environment %theme
  }{%
    Spring Semester 2024 %project period
  }{}{%
    %list of group members
    Rikke Udengaard
  }{%
    %list of supervisors
    Rocio Rodriguez Cano \\
    Jan H. Mikkelsen
  }{%
    1 % number of printed copies
  }{%
    \today % date of completion
  }%
}{%department and address
  \textbf{Electronics and IT}\\
  Aalborg University\\
  \href{http://www.aau.dk}{www.aau.dk}
}{% the abstract
  In this project two directional antennas, one fixed and one steerable, are employed to determine the direction of transmission in the horizontal plane and rotate to face that direction. To test the full system with and without a static intruder two identical horn antennas were used. Each antenna has a gain of \SI{11.41}{\decibel} at $f=\SI{4.75}{\giga\hertz}$ and \SI{12.99}{\decibel} at $f=\SI{5.65}{\giga\hertz}$. The results show that the maximum gain of the received signal at both frequencies is in the direction of the transmitter with reflections detected in directions of vertical surfaces in the measurement space, and that the maximum gain in the direction of the transmitter measured is higher without intruder. It is concluded that an intruder can be detected with the system when comparing with measurements without intruder and that the intruder affects the direction of the maximum gain measured in the environment. 
}

%The results indicate that at $f=\SI{5.65}{\giga\hertz}$ the received signal in the direction of the transmitter with intruder is still larger than the reflections. 