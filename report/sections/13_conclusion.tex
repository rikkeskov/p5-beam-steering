\chapter{Conclusion}
The turntable \textit{HEAD Acoustics HRT I} is a mechanical turntable controlled that can turn \SI{360}{\degree} in the horizontal plane and it is controlled in a \textit{Python}implemented control program, together with a \textit{Rohde and Schwarz ZVB8} vector network analyzer. The choices of external devices ensures that the system requirements are met and the implementation, with the \textit{Python threading} module, of the device control and communication with concurrency allows for efficient measuring. A number of possible antenna designs for a directive antenna can be used, however, two identical horn antennas where chosen because the chosen horn antenna type has a high directivity and matching polarization. The horn antennas are designed for \SI{4.4}{\giga\hertz} to \SI{7}{\giga\hertz} and have $S_{11}$-parameter of \SI{-19.56}{\decibel} and maximum gain of \SI{12.99}{\decibel} at \SI{5.65}{\giga\hertz} and $S_{11}$-parameter of \SI{-15.71}{\decibel} and maximum gain of \SI{11.41} at \SI{4.75}{\giga\hertz}. These two frequencies are used for test of full system. The accept test shows that with a straight-line setup and with a corner-to-corner setup it is possible to see, in the comparisons with and without object blocking the line-of-sight, an intruder in the measurement area. The test also shows that the measurements at $f=\SI{5.65}{\giga\hertz}$ all have a higher maximum gain and that the receiver antenna is generally able to detect the transmitter even with an intruder. The intruder dampens the received signal significantly, which is shown in the gain difference with and without intruder. Therefore, it can be concluded that the system can be used to detect an intruder when transmitting at both \SI{4.75}{\giga\hertz} and \SI{5.65}{\giga\hertz} when the transmitter faces the receiver. 