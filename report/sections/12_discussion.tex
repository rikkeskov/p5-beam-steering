\chapter{Discussion}
The aim of this project as set out in the introduction is to steer an antenna beam to detect a transmitter and focus the beam there, and also to detect an intrusion between the transmitter and the receiver. The turntable allows for mechanical beam steering in horizontal plane. This limits the ability to detect intruders to the horizontal plane, centered at the level of the antenna, meaning the setup for example would be unable to detect drones or small remote-controlled vehicles. The setup will not be able to detect intrusion above or below this level of the antenna, limiting the functionality of the setup. The control program with the electrically controlled mechanical turntable allows for the setup to be entirely automated. To have efficient intruder detection an automated setup must be a minimal requirement which the tests does show is possible.
\\[12pt]
The accept test was performed at the two frequencies \SI{4.75}{\giga\hertz} and \SI{5.65}{\giga\hertz}. There are clear differences between each frequency in all tests, showing that the choice of frequency affects the result. Other frequencies could have been chosen, as seen in the test of $S_{11}$ in section \ref{s:sparam_test}, that have a lower reflection coefficient than \SI{4.75}{\giga\hertz}. At \SI{5.65}{\giga\hertz} the tests show that even with an intruder between the transmitter and receiver, the receiver can locate the general direction of the transmitter and not only reflections in the measurement area. However, the gain in the direction of the transmitter is significantly reduced, therefore, when comparing the measured gain with and without intruder the setup could be expanded to use this comparison to alert of an intruder. The measured gain is also generally higher at \SI{5.65}{\giga\hertz} than at \SI{4.75}{\giga\hertz} at all angle positions. A more directive antenna could improve the setup by giving higher granularity of the position steps allowing for more accurate intruder location detection and by giving larger gain measurements for better differentiation at the angle steps. The latter, of which is also achieved with the horn antenna resonance frequency at \SI{5.65}{\giga\hertz} rather than \SI{4.75}{\giga\hertz}.
\\[12pt]
The test with the antennas perpendicular to each other shows that the layout of the test space greatly affects the measurements by giving different reflections at different positions and that with the setup of a transmitter antenna and receiver antenna, the result is best with the antennas facing each other. This dual-antenna setup adds another requirement to the setup, which is that the antennas must have matching polarization and function at the same frequency. In this case, choosing two identical horn antennas is a solution. 
\\[12pt]
The control program has been written in \textit{Python} using the \textit{threading} module in order to achieve concurrency. The I/O operations of the communication with the turntable and the VNA allow for some wait time to be exploited when using \textit{threading}, but the control program does also contain computational statements, such as appending and comparing data variables and calculating the maximum gain position. These must be performed sequentially and therefore do not benefit from concurrency. Further, the control program is not implemented to run endlessly, and will stop after calculating the maximum gain position. To achieve intruder detection alertion, the program should run indefinitely and be able to show an alarm. In the case of sending an alarm, the concurrency with the \textit{threading} module can be used further to make the control program be able to continue turning and measuring while reporting data or alerting a user of intrusion.

%The turntable chosen for the project, the \textit{HEAD Acoustics Remote-operated Turntable, model HRT I 6498} uses a step motor. In the setup the load on the table is constant and the position of the antenna must be repeated between each test to ensure the gains at each position can be compared. The mechanical turning does force the cable attached to the antenna to turn with it. This introduces some error to the measurements as the phase of the received signal become distorted. A better solution would be to electrically control the beam such as to avoid turning the antenna.