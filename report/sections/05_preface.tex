\chapter*{Preface\markboth{Preface}{Preface}}
\addcontentsline{toc}{chapter}{Preface}

%A preface is your opportunity to inform your audience about your experiences during the writing of your thesis or dissertation.

%A preface is much more personal than other types of academic writing. It is written mainly in the first person and is one of the few places where using first-person is considered acceptable. Your preface is typically written upon completion of your thesis or dissertation, as a final step.

% The following items can be included in the preface:

%     Your (brief) personal background
%     Any (brief) personal experiences or circumstances that motivated you to pursue this type of academic work
%     The target group for which your thesis or dissertation was written
%     Your name, the place name, and the date at the time of writing, at the end of the preface

% While it’s common practice to briefly acknowledge any individuals and/or institutions who have helped you during your writing and editing process, these should mostly be saved for your acknowledgements section.

% Indeed, it is often common practice to write either a preface or an acknowledgements section, not both.

This project was developed by Rikke Udengaard in the Spring semester of 2024 at Department of Electronic System at Aalborg University. The theme of this 5th semester project is \textit{Digital and Analogue Systems Interacting with the Environment}. 
\\[12pt]
\noindent The final product includes source code that is available on \textit{Github}. The link to the repository can be found in appendix \ref{a:code} with some of the source code is explained in chapter \ref{ch:implementation}. The repository also contains the test result logs which are also available in condensed form in the accept test results section \ref{s:accept_test}.
\\[12pt]
\noindent The block diagrams used throughout this report have been generated with \url{app.diagrams.net/}. \textit{CST Studio} and \textit{Matlab} have been used by the author to generate the remaining figures. The sources for the figures are listed in the figure captions. 
\\[12pt]
\noindent Prefixes and units are written in accordance to the SI system of units. The bibliography is found after the main report and before the appendix. The references are numbered in althebetical order according to the author's surname. 
\\[12pt]
\noindent The author would like to thank Kim Olesen for his assistance in the antenna lab at Aalborg University including measurements of the antenna. The author would also like to thank Rocio Rodriguez Cano for her active role as supervisor during the project.

\vspace{\baselineskip}\hfill Aalborg University, \today
\vfill\noindent
\vspace{3\baselineskip}
\begin{center}
\begin{minipage}[b]{0.45\textwidth}
 \centering
 \rule{\textwidth}{0.5pt}
  Rikke Udengaard\\
 {\footnotesize <rudeng20@student.aau.dk>}
\end{minipage}
\end{center}