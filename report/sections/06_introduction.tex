\chapter{Introduction}
Modern requirements for higher data speed from users all over the world has introduced new developments in antenna design for wireless communication, specifically with 5G and beyond. The data speeds needed for fast and immediate connectivity requires the antennas to transmit much more power than previously. This has given way for new antenna designs that have a higher gain and also higher directivity~\cite{beamsteering}. This also leaves areas where the antenna transmits very little and therefore areas where, if a user is positioned here, there will be very slow data rates. This can be mitigated by placing directive antennas in an array to ensure that the radiation pattern of the antennas covers the needed area. Doing this, the owner is sure that no matter where the user is in the needed area, there will always be signal, but it is also an expensive solution requiring a lot of hardware to cover larger physical areas. Another solution to mitigate the slow data rates in poorly covered areas is by controlling the beam of the antenna or antenna array and turning it to where the user is~\cite{beamsteering}. This means that fewer antennas could cover a larger area. 

Some of the beam steering designs include use of antenna arrays and beam forming to multiply and magnify the electromagnetic waves. It is also possible with phased antenna arrays to steer the direction of the beam by changing the signal phase to each antenna element and thereby electronically controlling the direction of the beam towards the user~\cite{beamsteering}. In general, being able to steer the beam of a directive antenna in a direction ensures that a larger area of operation is possible although not concurrently.

Steering the beam towards the user makes it possible to establish a line-of-sight from the radio transmitter to the receiving user, which is the ideal transmission situation. Sometimes, however, the user might be moving or the line-of-sight path to the user might be broken by another object that interferes. This in itself leaves a new problem for the data transmission, but it can also be used to detect whether or not a physical object is intruding in the area between the receiver and the transmitter. Physical intruder detection systems are used to detect unauthorised entry into protected areas and usually include sensors and alarms of different kinds~\cite{ids}. Intruder detection systems are used for many applications ranging from protecting homes to military facilities and therefore exist in many configurations for the different application types and users.

This project has focus on beam steering of an antenna both for detecting a user and focusing the beam towards the user but also to detect an intrusion between a transmitter and a receiver.