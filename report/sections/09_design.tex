\chapter{Design of Antenna Beam Steering} 
\todo[inline,color=red]{short introduction}
% Kunne konstruere systemer der består af et eller flere computerbaserede systemer indlejret i fysiske omgivelser, der involverer transformation mellem analoge og digitale signaler således at en a priori defineret opførsel af systemet bliver opnået.
% Kunne anvende feedback til at reducere påvirkningen af forstyrrelser, usikkerheder etc. samt kunne opstille krav til en ønsket systemrespons for lineære systemer og opnå denne.
% Kunne opstille krav til realtidskommunikation mellem computersystemer og implementere dette.

\section{Antenna Design}
\todo[inline,color=red]{explain design of horn antenna and why}
% Kunne forstå behandling og udveksling af analoge og digitale signaler mellem (del-)systemer, herunder frekvensresponser, fase- og gainkarakteristikker, sampling, analog og digital filtrering etc.
% Have grundlæggende viden om trådløs kommunikation. 

%In order to prevent aliasing the alias-free range must be calculated. As described in section \ref{sss:aliasing} the range depends on the settings on the network analyzer. The frequency span together with the number of points also affect how fast each measurement of the entire span can be made. 
%\todo[inline, color=red]{describe what happens with each setting + what sweep frequency(?) does}
%The frequency span is chosen from \SI{5}{\giga\hertz} to \SI{7}{\giga\hertz} which is most of the range of which the horn antenna is designed to work in. The number of points is chosen to be \SI{1001}{} because lowering this value by a couple factors will affect the measurement speed negatively. Therefore the alias-free range is
%\begin{equation}
%    Range = \frac{1001-1}{\SI{5E9}{\hertz}-\SI{7E9}{\hertz}} \SI{300E6}{\meter\per\second} = 150
%    \tagaddtext{[\si{\meter}]}
%\end{equation}

\section{Choice of Turntable}
\todo[inline,color=red]{explain choice of turntable and how it meets reqs}
% Kunne forstå digitale og analoge overføringsfunktioner beskrevet via hhv. z-operatoren og Laplace-operatoren, herunder elementer som poler, nulpunkter, analoge og digitale implementeringer, overføringsfunktionsmatricer etc.
% kunne linearisere ikke-lineære systemmodeller med henblik på at approksimere dem med lineære modeller.
% Kunne anvende metoder til at modellere fysiske systemer, herunder elektriske, elektromekaniske, termiske og hydrodynamiske systemer, på et niveau hvor modellerne kan bruges i designet af elektroniske systemer der interagerer med deres omgivelser.

\section{Choice of Vector Network Analyzer}
\todo[inline,color=red]{explain choice of vna and how it meets reqs}
% Have kendskab til forskellige metoder til at designe analoge og digitale filtre
% Kunne anvende Fouriertransformationen til at analysere digitale signaler.


\section{Design of Software Control Program}
\todo[inline,color=red]{explain design principles of software}
% Have viden om teorier og metoder til spektralestimering. 
% HVILKET BETYDER:
% Anvende relevante værktøjer som f.eks. Matlab eller Python til spektralestimering.
% Teori og metoder til non-parametrisk spektralestimering, herunder f.eks. Diskret Fourier Transformation (DFT) og dennes realisation i form af Fast Fourier Transformation (FFT) og Short Time Fourier Transformation (STFT).
% Kunne anvende relevante softwareværktøjer til at simulere ovenstående systemer.
The control of the turntable and VNA is programmed in Python. The control of the turntable is implemented as adviced by the manual (see \cite{hrt_i_manual}), with the supplied software \textit{RC-HRT I}. However, in order to control the turntable concurrently with the VNA, the \textit{RC-HRT I} software is controlled via Python with the module \textit{pywin32}, which provides access to the Microsoft Windows APIs~\cite{pywin32} and therefore makes the \textit{RC-HRT I} software controllable in Python. \textit{Rohde \& Schwarz} have published a Python module for control of the VNA which is used. Finally, thread parallelism is achieved with the module \textit{threading}.
%here more about the "main" control software 

The communication with the turntable and the VNA are handled internally in the Python modules that instead provide a functional, high-level interface. However, the communication protocols and interfaces that the turntable and VNA require, do affect the design of the control software. %More about why this is

\subsection{Communication Interfaces}
\todo[inline,color=red]{explain communication to and from vna and tt in depth}
% Have viden om OSI netværksmodellen.
The communication interfaces of the program is limited by the choice of the turntable and VNA. The \textit{HEAD Acoustics, HRT I} turntable that has been chosen can be controlled by serial communication using the RS485 standard~\cite{hrt_i_data_sheet}. The \textit{R\&S ZVB8} VNA has a LAN type interface for control and data transfer and ??~\cite{vna_data_sheet_descrip}.
% Have viden om protokoller i forskellige lag i systemer der kommunikerer med andre systemer.
% Kunne anvende metoder til at konstruere distribuerede systemer ved brug af kommunikationshardware, multi-programming og basale netværksprotokoller.


\subsection{Concurrency}
% kunne forstå realtidsaspekter i forhold til digitale systemer der kommunikerer med andre analoge og/eller digitale systemer.
As previously written, the turntable is controlled via serial communication while the data from the VNA is read via a network cable and TCP. Python serial communication and the communication with the VNA are both I/O bound tasks. Therefore, in order to achieve real-time communication and data processing the wait time of the I/O bound operations is exploited by use of the Python module \textit{Threading}. \textit{Threading} uses a single processor and pre-emptive multitasking to acheive concurrency~\cite{concurrency}. 
\todo[inline,color=red]{explain how program runs}

% band width
% channel base power = 10 dbm but what is max?
% aliasing
% invers fft
% s params ?