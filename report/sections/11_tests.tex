\chapter{Measurement Results} \label{s:accept_test}
This chapter includes test descriptions and results to validate the design against the requirements. First, the horn antennas is tested to find the gain and $S_{11}$-parameter. The descriptions and results of those tests can be found in appendix \ref{s:sparam_test} and \ref{s:rad_test}. Finally, the full test of the beam steering and gain measuring is tested with the transmitter fixed at a known location in the test area. The aim of this test is to determine the full functionality of the developed product. The test should show that the receiver antenna on the turntable is able to scan the test area and measure the received power at fixed angles, before selection the location with the maximum received power and focusing its beam on that location. The test also contains test of intruder detection. In these scenarios the line-of-sight between the transmitter and receiver antennas is broken by an object. Figure \ref{fig:experiment-setup} shows a simple diagram of the test setup from a horizontal view.

\begin{figure}[H]
    \centering
    \includegraphics[width=0.8\textwidth]{figures/accept_test_setup.pdf}
    \caption{Setup for test of functionality of beam steering device.} \label{fig:experiment-setup}
\end{figure}

\section{Equipment}
To perform the test, the following equipment is needed:

\begin{itemize}
    \item \textit{HEAD Acoustics Remote-operated Turntable, model HRT I 6498} with \SI{24}{\volt} DC \SI{60}{W} power supply
    \item D-sub 9-pin to USB-A cable to connect turntable to PC
    \item PC with one USB-A port and one LAN port
    \item \textit{Rohde \& Schwarz ZVB8} Vector Network Analyzer
    \item Network cable (8-pin RJ-45 connector) to connect VNA to PC
    \item Two identical horn antennas with dimensions as seen on figure \ref{fig:horn_design} in section \ref{s:ant_design}
    \item Two \SI{50}{\ohm} antenna cables
    \item Two identical horn antennas as DUTs.
\end{itemize}

Moreover, the test must be performed in a controlled environment in order to ensure that the turntable and VNA can function optimally. The temperature must not be below \SI{5}{\celsius} or above \SI{40}{\celsius} with a relative humidity in the range \SI{20}{\percent} - \SI{80}{\percent}~\cite{hrt_i_data_sheet}\cite{vna_data_sheet_spec}.

\section{Procedure}
The following steps outline how to perform the test:

\begin{enumerate}
    \item Power ZVB8 and HRT I. 
    \item Calibrate ZVB8 with calibration unit.
    \item Connect Windows PC to ZVB8 and HRT I.
    \item Connect transmitter antenna to port 2 on ZVB8 with antenna cable. Set channel power level to \SI{10}{dBm}.
    \item Connect receiver antenna to port 1 on ZVB8 with antenna cable.
    \item Setup transmitter antenna to point in a direction defined by the setup configuration. 
    \item Mount the receiver antenna on HRT I.
    \item Load \textit{Python} code on Windows PC and run control program.
\end{enumerate}

Repeat the test procedure for different setups; antennas pointing straight to each other with and without intruder person, with the antennas in either corner of a room with and without intruder table, and with the transmitter perpendicular to the receiver antenna. 

\section{Test Results}
The tables in the following sections include condensed versions of the logs printed with every test of the full setup. The logs can be found through the link in appendix \ref{a:code}. 

\subsection{Straight Setup}
Thia test scenario is made with the antennas pointing directly towards each other without intruder. The distance between the antennas is $d=\SI{5.4}{\meter}$. The start position is \SI{10}{\degree}, the end position is \SI{150}{\degree} and the increase is \SI{20}{\degree}. The following figure \ref{fig:a2_1} shows the setup:
\begin{figure}[H]
    \centering
    \includegraphics[width=0.7\textwidth]{figures/test_los_straight.JPG}
    \caption{View from receiver antenna at position \SI{50}{\degree} towards transmitter antenna.} \label{fig:a2_1}
\end{figure}

The test results can be found in table \ref{tab:a2_1a} and \ref{tab:a2_1b}.
\begin{table}[H]
    \centering
    \begin{tabular}{l|l|l|l}
        \multicolumn{4}{l}{\textbf{Frequency = 4.75 GHz}}         \\
        \hline
        \textbf{Position} & \multicolumn{3}{l}{\textbf{Power Measurement (dB)}} \\
        \textbf{(degrees)}  & Test 1    & Test 2  & Test 3  \\
        \hline
        \hline
        10      & -51.33    & -51.15    & -51.01 \\
        30      & \textcolor{red}{-46.82}    & \textcolor{red}{-46.63}    & \textcolor{red}{-46.52} \\
        50      & -46.96    & -46.77    & -46.66 \\
        70      & -49.07    & -48.88    & -48.78 \\
        90      & -55.30    & -55.18    & -55.07 \\
        110     & -70.33    & -69.99    & -69.81 \\
        130     & -67.82    & -67.30    & -67.27 \\
        150     & -66.85    & -66.73    & -66.58
        \end{tabular}
    \caption{Table of power measurements at each position repeated three times at frequency $f=\SI{4.75}{\giga\hertz}$. The maximum gain of each test is highlighted in red.}
    \label{tab:a2_1a}
\end{table}

\begin{table}[H]
    \centering
    \begin{tabular}{l|l|l|l}
        \multicolumn{4}{l}{\textbf{Frequency = 5.65 GHz}}         \\
        \hline
        \textbf{Position} & \multicolumn{3}{l}{\textbf{Power Measurement (dB)}} \\
        \textbf{(degrees)}  & Test 1    & Test 2  & Test 3  \\
        \hline
        \hline
        10      & -45.96    & -46.04    & -45.99 \\
        30      & -34.88    & -34.81    & -34.76 \\
        50      & \textcolor{red}{-31.05}    & \textcolor{red}{-30.99}    & \textcolor{red}{-30.93} \\
        70      & -34.61    & -34.56    & -34.51 \\
        90      & -49.59    & -49.54    & -49.47 \\
        110     & -64.89    & -64.79    & -64.67 \\
        130     & -50.16    & -50.15    & -50.15 \\
        150     & -58.68    & -58.68    & -58.79
        \end{tabular}
    \caption{Table of power measurements at each position repeated three times at frequency $f=\SI{5.65}{\giga\hertz}$. The maximum gain of each test is highlighted in red.}
    \label{tab:a2_1b}
\end{table}

Comparing the results of the direct, straight line-of-sight without intruder between $f=\SI{4.75}{\giga\hertz}$ and $f=\SI{5.65}{\giga\hertz}$ it is seen that the maximum gain is not in the same direction but instead at \SI{30}{\degree} at $f=\SI{4.75}{\giga\hertz}$ and at \SI{50}{\degree} at $f=\SI{5.65}{\giga\hertz}$. Looking at the data for $f=\SI{4.75}{\giga\hertz}$, it can be seen that the difference in magnitude from \SI{30}{\degree} to \SI{50}{\degree} is very small (<\SI{0.14}{\decibel}) where its larger (>\SI{3.82}{\decibel}) for $f=\SI{5.65}{\giga\hertz}$ for the same angle step. This indicates that at $f=\SI{5.65}{\giga\hertz}$ the beam of the horn antenna is narrower. The maximum gain is also larger at $f=\SI{5.65}{\giga\hertz}$ at averagely \SI{30.99}{\decibel} compared to the gain at $f=\SI{4.75}{\giga\hertz}$ at averagely \SI{-46.66}{\decibel}.

\subsection{Straight Setup With Intruder}
The test scenario \textit{straight with person intruder} is made with the antennas pointing directly towards each other. The distance between the antennas is $d=\SI{5.4}{\meter}$ and the object, the person, is placed \SI{2}{\meter} from the transmitter and \SI{3.4}{\meter} from the receiver. The start position is \SI{10}{\degree}, the end position is \SI{150}{\degree} and the increase is \SI{20}{\degree}. The following figure \ref{fig:a2_5} shows the setup:
\begin{figure}[H]
    \centering
    \includegraphics[width=0.7\textwidth]{figures/test_intruder_person.JPG}
    \caption{View from receiver antenna at position \SI{50}{\degree} towards transmitter antenna. The green arrow points towards the location the where the person is.} \label{fig:a2_5}
\end{figure}

The test results can be found in table \ref{tab:a2_5a} and \ref{tab:a2_5b}.
\begin{table}[H]
    \centering
    \begin{tabular}{l|l|l|l}
        \multicolumn{4}{l}{\textbf{Frequency = 4.75 GHz}}         \\
        \hline
        \textbf{Position} & \multicolumn{3}{l}{\textbf{Power Measurement (dB)}} \\
        \textbf{(degrees)}  & Test 1    & Test 2  & Test 3  \\
        \hline
        \hline
        10      & \textcolor{red}{-53.88}    & \textcolor{red}{-51.03}    & \textcolor{red}{-53.15} \\
        30      & -65.65    & -62.04    & -58.07 \\
        50      & -60.18    & -60.21    & -63.31 \\
        70      & -60.91    & -61.45    & -63.88 \\
        90      & -68.03    & -64.38    & -67.73 \\
        110     & -78.76    & -67.41    & -85.01 \\
        130     & -68.04    & -86.78    & -72.53 \\
        150     & -65.90    & -66.63    & -64.46
        \end{tabular}
    \caption{Table of power measurements at each position repeated three times at frequency $f=\SI{4.75}{\giga\hertz}$. The maximum gain of each test is highlighted in red.}
    \label{tab:a2_5a}
\end{table}

\begin{table}[H]
    \centering
    \begin{tabular}{l|l|l|l}
        \multicolumn{4}{l}{\textbf{Frequency = 5.65 GHz}}         \\
        \hline
        \textbf{Position} & \multicolumn{3}{l}{\textbf{Power Measurement (dB)}} \\
        \textbf{(degrees)}  & Test 1    & Test 2  & Test 3  \\
        \hline
        \hline
        10      & -48.58    & \textcolor{red}{-44.45}    & -44.61 \\
        30      & -49.91    & -55.05    & -49.71 \\
        50      & \textcolor{red}{-43.50}    & -45.52    & \textcolor{red}{-39.29} \\
        70      & -44.11    & -45.39    & -40.85 \\
        90      & -54.07    & -62.36    & -50.56 \\
        110     & -63.50    & -74.10    & -59.50 \\
        130     & -52.11    & -53.01    & -52.56 \\
        150     & -60.54    & -57.75    & -58.38
        \end{tabular}
    \caption{Table of power measurements at each position repeated three times at frequency $f=\SI{5.65}{\giga\hertz}$. The maximum gain of each test is highlighted in red.}
    \label{tab:a2_5b}
\end{table}

The intruder changes the ability of the control program to correctly identify the direction of the transmitter antenna. At $f=\SI{4.75}{\giga\hertz}$ the maximum gain is at \SI{10}{\degree} which is in the direction of the close wall on the right-hand side when facing the transmitter. 

Further, the test data (seen in table \ref{tab:a2_5a}) also show that the gain at \SI{50}{\degree} and \SI{70}{\degree} is averagely higher than at other angle steps, indicating that the receiver antenna still can detect the transmitter with a person intruder, but that the intruder does affect the received signal. At $f=\SI{5.65}{\giga\hertz}$ the data is not consistent across all three tests. The receiver antenna receives the maximum gain in the same direction as the transmitter antenna in the line-of-sight test scenario. 

When comparing to the same setup without intruder, where the maximum gain is also at \SI{50}{\degree}, it can be seen that the intruder does not dampen the signal to a level, where the reflection signal are larger than the direct, transmitted signal. Comparing the values, the maximum gain measured with an intruder is averagely $\SI{42.41}{\decibel}-\SI{30.99}{\decibel}=\SI{11.42}{\decibel}$ less than without intruder. In this scenario the frequency $f=\SI{4.75}{\giga\hertz}$ can be a better frequency choice.

\subsection{Corner-To-Corner Setup}
This test is performed with the antennas pointing directly towards each other from a skew angle in each their own corner of the room. The distance between the antennas is not measured. The start position is \SI{10}{\degree}, the end position is \SI{150}{\degree} and the increase is \SI{20}{\degree}. The following figure \ref{fig:a2_2} shows the setup:
\begin{figure}[H]
    \centering
    \includegraphics[width=0.7\textwidth]{figures/test_los_corner.JPG}
    \caption{View from receiver antenna at position \SI{10}{\degree} towards transmitter antenna.} \label{fig:a2_2}
\end{figure}

The test results can be found in table \ref{tab:a2_2a} and \ref{tab:a2_2b}.
\begin{table}[H]
    \centering
    \begin{tabular}{l|l|l|l}
        \multicolumn{4}{l}{\textbf{Frequency = 4.75 GHz}}         \\
        \hline
        \textbf{Position} & \multicolumn{3}{l}{\textbf{Power Measurement (dB)}} \\
        \textbf{(degrees)}  & Test 1    & Test 2  & Test 3  \\
        \hline
        \hline
        10      & -59.79    & -59.20    & -59.13 \\
        30      & -52.45    & -52.08    & -52.00 \\
        50      & -53.22    & -52.94    & -52.86 \\
        70      & \textcolor{red}{-48.34}    & \textcolor{red}{-48.11}    & \textcolor{red}{-48.06} \\
        90      & -51.32    & -51.09    & -51.04 \\
        110     & -70.33    & -58.41    & -58.33 \\
        130     & -75.39    & -75.27    & -75.29 \\
        150     & -62.40    & -62.15    & -62.07
        \end{tabular}
    \caption{Table of power measurements at each position repeated three times at frequency $f=\SI{4.75}{\giga\hertz}$. The maximum gain of each test is highlighted in red.}
    \label{tab:a2_2a}
\end{table}

\begin{table}[H]
    \centering
    \begin{tabular}{l|l|l|l}
        \multicolumn{4}{l}{\textbf{Frequency = 5.65 GHz}}         \\
        \hline
        \textbf{Position} & \multicolumn{3}{l}{\textbf{Power Measurement (dB)}} \\
        \textbf{(degrees)}  & Test 1    & Test 2  & Test 3  \\
        \hline
        \hline
        10      & -45.73    & -45.41    & -45.37 \\
        30      & -43.94    & -43.82    & -43.79 \\
        50      & -34.21    & -34.01    & -33.96 \\
        70      & \textcolor{red}{-32.29}    & \textcolor{red}{-32.15}    & \textcolor{red}{-32.10} \\
        90      & -37.35    & -37.27    & -37.23 \\
        110     & -51.65    & -51.62    & -51.63 \\
        130     & -68.07    & -68.20    & -68.02 \\
        150     & -56.72    & -56.73    & -56.66
        \end{tabular}
    \caption{Table of power measurements at each position repeated three times at frequency $f=\SI{5.65}{\giga\hertz}$. The maximum gain of each test is highlighted in red.}
    \label{tab:a2_2b}
\end{table}

The test data with the corner-to-corner scenario shows the gain when the antennas are placed further apart in a corner-to-corner configuration with and without a table with a large, square electronic box as intruder. As seen on figure \ref{fig:a2_2} the transmitter antenna is moved to the left, meaning that the turntable must turn further in order to face the receiver antenna towards the transmitter antenna. This reflects in the results in table \ref{tab:a2_2a} and \ref{tab:a2_2b} where both at $f=\SI{4.75}{\giga\hertz}$ and $f=\SI{5.65}{\giga\hertz}$ the maximum gain is at \SI{70}{\degree}. Similarly as with the straight line-of-sight setup (seen in figure \ref{fig:a2_4}) the maximum gain is larger at the higher frequency. 

\subsection{Corner-To-Corner Setup With Intruder}
The test scenario \textit{corner-to-corner with table intruder} is made with the transmitting antenna pointing in the direction of the receiver antenna, with the antennas each in an opposite corner. The distance between the antennas is not measured. The start position is \SI{10}{\degree}, the end position is \SI{150}{\degree} and the increase is \SI{20}{\degree}. The following figure \ref{fig:a2_4} shows the setup:
\begin{figure}[H]
    \centering
    \includegraphics[width=0.7\textwidth]{figures/test_intruder_table.JPG}
    \caption{View from receiver antenna at position \SI{70}{\degree} towards transmitter antenna.} \label{fig:a2_4}
\end{figure}

The test results can be found in table \ref{tab:a2_4a} and \ref{tab:a2_4b}.
\begin{table}[H]
    \centering
    \begin{tabular}{l|l|l|l}
        \multicolumn{4}{l}{\textbf{Frequency = 4.75 GHz}}         \\
        \hline
        \textbf{Position} & \multicolumn{3}{l}{\textbf{Power Measurement (dB)}} \\
        \textbf{(degrees)}  & Test 1    & Test 2  & Test 3  \\
        \hline
        \hline
        10      & \textcolor{red}{-57.75}    & \textcolor{red}{-57.52}    & \textcolor{red}{-57.40} \\
        30      & -59.87    & -59.68    & -59.57 \\
        50      & -65.42    & -65.22    & -65.14 \\
        70      & -62.53    & -62.40    & -62.26 \\
        90      & -59.81    & -59.60    & -59.49 \\
        110     & -64.94    & -64.81    & -64.74 \\
        130     & -68.94    & -68.80    & -68.63 \\
        150     & -62.11    & -61.96    & -61.85
        \end{tabular}
    \caption{Table of power measurements at each position repeated three times at frequency $f=\SI{4.75}{\giga\hertz}$. The maximum gain of each test is highlighted in red.}
    \label{tab:a2_4a}
\end{table}

\begin{table}[H]
    \centering
    \begin{tabular}{l|l|l|l}
        \multicolumn{4}{l}{\textbf{Frequency = 5.65 GHz}}         \\
        \hline
        \textbf{Position} & \multicolumn{3}{l}{\textbf{Power Measurement (dB)}} \\
        \textbf{(degrees)}  & Test 1    & Test 2  & Test 3  \\
        \hline
        \hline
        10      & -42.72    & -42.68    & -42.61 \\
        30      & -47.23    & -47.15    & -47.10 \\
        50      & \textcolor{red}{-35.61}    & \textcolor{red}{-35.55}    & \textcolor{red}{-35.49} \\
        70      & -35.83    & -35.75    & -35.69 \\
        90      & -41.75    & -41.68    & -41.61 \\
        110     & -54.32    & -54.25    & -54.20 \\
        130     & -77.38    & -76.43    & -76.11 \\
        150     & -57.60    & -57.49    & -57.27
        \end{tabular}
    \caption{Table of power measurements at each position repeated three times at frequency $f=\SI{5.65}{\giga\hertz}$. The maximum gain of each test is highlighted in red.}
    \label{tab:a2_4b}
\end{table}

The test is performed again with a table as an intruder. At $f=\SI{4.75}{\giga\hertz}$ the maximum gain is at position \SI{10}{\degree} which is not in the direction of the transmitting antenna but rather towards the right-hand wall when viewing from the receiver into the measurement space. This indicates that the receiver antenna receives reflections from the wall surface rather than directly from the transmitter. At $f=\SI{5.65}{\giga\hertz}$ this changes however, and the maximum gain is averagely \SI{35.55}{\decibel} at position \SI{50}{\degree}. 

The gain is approximately the same at position \SI{70}{\degree} as at position \SI{50}{\degree} with averagely difference of \SI{0.21}{\decibel}, which is the direction of the transmitter antenna. This indicates that even with an intruder, the receiver is able to locate the transmitter when at the frequency $f=\SI{5.65}{\giga\hertz}$. Comparing with the test without intruder, it can be seen at \SI{70}{\degree} there is a significant difference in what is measured in both tests and a comparable result at \SI{50}{\degree}. This indicates that the table intruder dampens the signal at \SI{70}{\degree} exactly and that the intruder can be detected when comparing the two signals at \SI{70}{\degree}. In this scenario the frequency $f=\SI{5.65}{\giga\hertz}$ can be a better frequency choice.

\subsection{Perpendicular Antennas}
The test scenario is made with the transmitting antenna pointing perpendicular to the receiver antenna. The distance between the antennas is $d=\SI{5.4}{\meter}$. The start position is \SI{10}{\degree}, the end position is \SI{150}{\degree} and the increase is \SI{20}{\degree}. The following figure \ref{fig:a2_3} shows the setup:
\begin{figure}[H]
    \centering
    \includegraphics[width=0.7\textwidth]{figures/test_los_perpendicular.JPG}
    \caption{View from receiver antenna at position \SI{50}{\degree} towards transmitter antenna.} \label{fig:a2_3}
\end{figure}

The test results can be found in table \ref{tab:a2_3a} and \ref{tab:a2_3b}.
\begin{table}[H]
    \centering
    \begin{tabular}{l|l|l|l}
        \multicolumn{4}{l}{\textbf{Frequency = 4.75 GHz}}         \\
        \hline
        \textbf{Position} & \multicolumn{3}{l}{\textbf{Power Measurement (dB)}} \\
        \textbf{(degrees)}  & Test 1    & Test 2  & Test 3  \\
        \hline
        \hline
        10      & -74.27    & -74.29    & -74.33 \\
        30      & -72.12    & -72.13    & -72.21 \\
        50      & -77.03    & -76.04    & -75.61 \\
        70      & -71.35    & -71.17    & -70.58 \\
        90      & -69.45    & -69.47    & -69.71 \\
        110     & \textcolor{red}{-68.42}    & \textcolor{red}{-68.28}    & \textcolor{red}{-68.15} \\
        130     & -79.83    & -80.25    & -79.30 \\
        150     & -75.04    & -75.25    & -75.23
        \end{tabular}
    \caption{Table of power measurements at each position repeated three times at frequency $f=\SI{4.75}{\giga\hertz}$. The maximum gain of each test is highlighted in red.}
    \label{tab:a2_3a}
\end{table}

\begin{table}[H]
    \centering
    \begin{tabular}{l|l|l|l}
        \multicolumn{4}{l}{\textbf{Frequency = 5.65 GHz}}         \\
        \hline
        \textbf{Position} & \multicolumn{3}{l}{\textbf{Power Measurement (dB)}} \\
        \textbf{(degrees)}  & Test 1    & Test 2  & Test 3  \\
        \hline
        \hline
        10      & -58.25    & -58.25    & -58.34 \\
        30      & -55.13    & -55.12    & -54.97 \\
        50      & \textcolor{red}{-49.13}    & \textcolor{red}{-49.11}    & \textcolor{red}{-49.08} \\
        70      & -52.44    & -52.40    & -52.43 \\
        90      & -50.17    & -50.10    & -50.10 \\
        110     & -49.77    & -49.73    & -49.81 \\
        130     & -52.26    & -52.20    & -52.26 \\
        150     & -58.72    & -59.04    & -58.64
        \end{tabular}
    \caption{Table of power measurements at each position repeated three times at frequency $f=\SI{5.65}{\giga\hertz}$. The maximum gain of each test is highlighted in red.}
    \label{tab:a2_3b}
\end{table}

The transmitter antenna is placed perpendicular to the receiver facing the right-hand wall when looking from the receiver to the transmitter in a straight line. The setup can be seen in figure \ref{fig:a2_3}. At $f=\SI{4.75}{\giga\hertz}$ the angle step with maximum gain is \SI{110}{\degree} which is not in the direction of the transmitter but instead in the opposite direction where the receiver antenna faces walls and tables with test equipment. This shows that the receiver antenna is unable to detect the transmitter at $f=\SI{4.75}{\giga\hertz}$ if the transmitter is facing perpendicular to the the receiver antenna position. 

At $f=\SI{5.65}{\giga\hertz}$ the receiver antenna correctly identifies the transmitter to be at angle step \SI{50}{\degree}. This indicates that at $f=\SI{5.65}{\giga\hertz}$ the gain of the transmitter antenna at the $\theta=\SI{90}{\degree}$ angle in the azimuth plane is so large, that the receiver still detects the signal instead of reflections from objects in the measurement space. 

\section{Test Results Comparison}
A further analysis of the test data can be made by comparing the measured gain at each position for all scenarios but excluding the perpendicular setup. The data is plotted in figure \ref{fig:gain_vs_pos_475} for $f=\SI{4.75}{\giga\hertz}$ and in figure \ref{fig:gain_vs_pos_565} for $f=\SI{5.65}{\giga\hertz}$. At $f=\SI{4.75}{\giga\hertz}$ the graph shows that the intruder clearly dampens the received signal in both scenarios. The test data at this frequency is not consistent across test scenarios and positions. 

The measured gain for test of the corner-to-corner with intruder increases in the position \SI{70}{\degree} to \SI{90}{\degree} where it decreases in the corner-to-corner without intruder test. This indicates that some signal is not blocked by the intruder at this position. The intruder in the straight test scenario dampens the measured gain from \SI{30}{\degree} to \SI{110}{\degree}. The decrease of the measured gain of the straight test with intruder is similar to the decreasing development of the measured gain without intruder as the position increases from \SI{30}{\degree} to \SI{110}{\degree}. 

The received signal present at \SI{110}{\degree} to \SI{150}{\degree} is not very consistent. This could indicate that there is unwanted signal disturbance or signal sources. The direction is far from the signal source, the transmitter antenna. In this direction reflections from vertical objects in the measurement space create reflections which can explain the increase in measured gain at \SI{150}{\degree}.

\begin{figure}[H]
    \centering
    \includegraphics[width=1\textwidth]{figures/gain_vs_pos_475.png}
    \caption{Measured gain at $f=\SI{4.75}{\giga\hertz}$ each position for the four scenarios straight or corner-to-corner and with or without intruder.} 
    \label{fig:gain_vs_pos_475}
\end{figure}

The tests performed at $f=\SI{5.65}{\giga\hertz}$ show a more clear trend when comparing each test scenario. The intruder clearly dampens the input throughout the entire position spectrum until the final position \SI{150}{\degree}, indicating that this position is in a direction too far away from the transmitter to be mainly influenced by the transmitted signal and is instead receiving a reflected signal. The measured gain in all four test scenarios is similar at this position. 

The difference between the measurement with and without intruder is largest for the straight setup. The intruder was in the straight scenario a person, whereas in the corner-to-corner scenario the intruder was a table. This indicates that the size of the intruder affects the ratio of measured gains. Finally, the difference in the placement of the transmitter antenna can be most clearly seen without intruder, where the maximum gain is at \SI{50}{\degree} in the straight test scenario and at \SI{70}{\degree} in the corner-to-corner test. 

\begin{figure}[H]
    \centering
    \includegraphics[width=1\textwidth]{figures/gain_vs_pos_565.png}
    \caption{Measured gain at $f=\SI{5.65}{\giga\hertz}$ each position for the four scenarios straight or corner-to-corner and with or without intruder.} 
    \label{fig:gain_vs_pos_565}
\end{figure}

The accept test show that at $f=\SI{5.65}{\giga\hertz}$ more power is received which is because, as shown in test of radiation pattern \ref{s:rad_test}, that the horn antenna has a higher gain at this frequency than at the lower frequency tested. 

While the turntable is turning the cable connected to the horn antenna will move with it. This will affect the phase of the signal. However, since the purpose of the test is to measure magnitude, this error is will not effect the test result. The test of the setup with a person as intruder is subject to measurement inaccuracies due to minimal movement.